%%%%%%%%%%%%%%%%%%%%%%%%%%%%%%%%%%%%%%%%%%%%%%%%%%%%%%%%%%%%%%%
%%  Versão final do IEEExplore
%%%%%%%%%%%%%%%%%%%%%%%%%%%%%%%%%%%%%%%%%%%%%%%%%%%%%%%%%%%%%%%

\documentclass[10pt, conference, compsocconf]{IEEEtran}
% \documentclass[a4paper]{sbgames}

\usepackage{times}
\usepackage{graphicx}
\usepackage{amsmath,amssymb,amsthm,siunitx}
\usepackage[brazil,american]{babel}
\usepackage[utf8]{inputenc}

%% use this for zero \parindent and non-zero \parskip, intelligently.
\usepackage{parskip}

%% the 'caption' package provides a nicer-looking replacement
%\usepackage[labelfont=bf,textfont=it]{caption}

\usepackage{url}

\begin{document}

%% Paper title.
\title{The Adventures of Lolo}


\author{\IEEEauthorblockN{Aquila Macedo Costa, Eduardo Ferreira Marques Cavalcante, and Matheus Cardoso de Souza}
\IEEEauthorblockA{Dept. of Computer Science\\
University of Brasilia\\ Brasilia, Brazil\\
Email: {costa.aquila@aluno.unb.br, marques.eduardo@aluno.unb.br, matheus-cardoso.mc@aluno.unb.br}}
}

 % \author{ Autor1
 %         \hspace{28pt} Autor2 \\
 %         \vspace{0pt} \\
 %         {University of Brasília, Dept. of Computer Science, Brazil} }

 % \contactinfo{autor1@gmail.com \\ autor2@hotmail.br }



%\teaser{
%  \includegraphics[width=\linewidth]{sample.pdf}
%  (a)\hspace{150pt} (b) \hspace{150pt}   (c)
%  \caption{(a) Guitar Hero III screen; (b) DE2-35 Kit on top of PlayStation 2; (c) Grybot}
%  \label{fig:01}
%}


% make the title area
\maketitle

% Abstract section.
\begin{abstract}
  It will be described in this article the process of the remake of the game The
  adventures of Lolo, developed pela HAL Laboratory para o Nintendo
  Entertainment System (NES) in 1989, using the low level programming language
  Assembly, fllowing the rules of the ISA (Instruction Set Architecture) RISC-V,
  assembled and executed in RARS. It will be detailed how the knowledge learned
  in Introduction to Computer Systems course, such as using the Bitmap Display
  and KDMMIOS, were implemented for the recreation of a customized game. Será
  descrito neste artigo o processo de recriação do jogo The adventures of

  % Lolo, desenvolvido pela HAL Laboratory para o Nintendo Entertainment System
  % (NES) em 1989, utilizando a linguagem de baixo nível Assembly, seguindo as
  % normas da ISA (Instruction Set Architecture) RISC-V, montado e executado no
  % RARS. Será detalhado como os conhecimentos aprendidos na disciplina Introdução
  % a Sistemas Computacionais, como o Bitmap Display e KDMMIOS, foram
  % implementados para a recriação do jogo de forma personalizada.
\end{abstract}

%% Keywords that describe your work.
% \keywords{Assembly, RISC-V, The Adventures of Lolo, Bitmap Display, RARS}
\begin{IEEEkeywords}
Assembly; RISC-V; The Adventures of Lolo; Bitmap Display; RARS
\end{IEEEkeywords}


\section{Introdução}
\label{sec:introducao}

O jogo \textit{The Adventures of Lolo} foi lançado em 1989 voltado, principalmente, para crianças, como um jogo de puzzle e raciocínio lógico, em que se deve solucionar variados problemas para passar de fase e resgatar a companheira de Lolo, Lala. O jogador controla Lolo, um ser azul, que possui poderes úteis para derrotar inimigos e solucionar os desafios de cada nível.

Como forma de aprendizado na disciplina Introdução a Sistemas Computacionais, foi proposto para os alunos a recriação do jogo The Adventures of Lolo, utilizando o RARS (RISC-V Assembler and Runtime Simulator). O projeto foi criado com o propósito de ensinar métodos de programação, raciocínio lógico e planejamento, além de introduzir os alunos à linguagem de baixo nível Assembly.

\begin{figure}[htb]
  \begin{center}
   \includegraphics[width=0.3\linewidth]{./Figures/capa_lolo_nes.png}
  \end{center}
  \caption{Capa do jogo no NES}
  \label{fig:01}
\end{figure}

\section{Metodologia}
\label{sec:Metodologia}

Para implementação do jogo no RARS, utilizaram-se principalmente as ferramentas
\textit{bitmap display}, o que permite imprimir as imagens (sprites) dos personagens,
blocos, inimigos entre outros, criando uma interface visual para o jogador, e o
KDMMIOS (Keyboard and Display MMIO Simulator), o qual possibilita enviar um dado
do teclado para o RARS, viabilizando a movimentação do Lolo por meio do teclado
do jogador.

De início, o projeto começou com a criação de um repositório no GitHub para fins
de organização do grupo e melhor versionamento de código, assim, partimos para
implementação da impressão de sprites no bitmap display. Para tal, precisamos
formular uma estratégia para renderização dos sprites.

\subsection{Arquitetura MIPS}{
\label{sec:MIPS}
Exemplo de subseção. A arquitetura MIPS \cite{patterson2005organizaccao} foi desenvolvida por ....


Lorem ipsum dolor sit amet, consectetur adipiscing elit. Aenean nec magna lectus. Donec egestas risus quis mollis venenatis. Nullam vel tellus enim. Aliquam erat volutpat. Phasellus a urna et tellus venenatis aliquet. Integer sit amet condimentum leo. Etiam massa nisi, rhoncus eget condimentum pharetra, imperdiet vitae eros. Nunc porta nisi facilisis, feugiat odio sit amet, ullamcorper tortor. Cras et urna cursus, mollis nibh quis, blandit risus.

Pellentesque tincidunt ultrices ex at varius. Pellentesque sit amet sapien in enim vehicula tincidunt vel ac augue. Integer ultricies vulputate massa at scelerisque. Curabitur ut quam porttitor, rhoncus felis eget, semper sapien. Integer accumsan nisi et leo suscipit iaculis. Suspendisse potenti. Class aptent taciti sociosqu ad litora torquent per conubia nostra, per inceptos himenaeos. Aenean pretium lacus sed quam volutpat congue. Ut sapien erat, dictum nec faucibus vel, finibus tempor nisl. 
}


\subsection{Simulador/Montador MARS}{
\label{sec:Mars}
Exemplo de subseção. O Mars \textit{MIPS Assembler and Runtime Simulator} \cite{Mars1} é um simulador desenvolvido por...

Lorem ipsum dolor sit amet, consectetur adipiscing elit. Aenean nec magna lectus. Donec egestas risus quis mollis venenatis. Nullam vel tellus enim. Aliquam erat volutpat. Phasellus a urna et tellus venenatis aliquet. Integer sit amet condimentum leo. Etiam massa nisi, rhoncus eget condimentum pharetra, imperdiet vitae eros. Nunc porta nisi facilisis, feugiat odio sit amet, ullamcorper tortor. Cras et urna cursus, mollis nibh quis, blandit risus.

Pellentesque tincidunt ultrices ex at varius. Pellentesque sit amet sapien in enim vehicula tincidunt vel ac augue. Integer ultricies vulputate massa at scelerisque. Curabitur ut quam porttitor, rhoncus felis eget, semper sapien. Integer accumsan nisi et leo suscipit iaculis. Suspendisse potenti. Class aptent taciti sociosqu ad litora torquent per conubia nostra, per inceptos himenaeos. Aenean pretium lacus sed quam volutpat congue. Ut sapien erat, dictum nec faucibus vel, finibus tempor nisl. 
}


\section{Resultados Obtidos}
\label{sec:Resultados}
Apresentar aqui os resultados obtidos, telas e link para vídeos e comentários.



Lorem ipsum dolor sit amet, consectetur adipiscing elit. Nunc eu mi cursus, pretium lectus vel, commodo neque. Nulla facilisi. Duis in quam non metus lobortis sagittis. Nunc ac auctor mi. Nullam imperdiet orci eget neque accumsan, eget ornare turpis cursus. Sed ut cursus sapien, vel accumsan neque. Donec dignissim maximus sapien non commodo. Praesent a gravida metus. Duis eget nulla luctus, finibus ex sed, ornare velit. Pellentesque habitant morbi tristique senectus et netus et malesuada fames ac turpis egestas. Curabitur placerat efficitur velit, eget auctor eros posuere nec. Morbi pretium purus in libero porttitor interdum.

Ut convallis egestas libero, sit amet tempor sem volutpat pretium. Nam ut libero mattis, interdum purus sit amet, fermentum diam. Integer a mi iaculis, egestas risus in, tempus eros. Donec elementum aliquet ante, nec ultrices lectus iaculis id. Integer pretium, mi id tempor blandit, elit sem viverra mi, sit amet rhoncus nunc sapien et mi. Etiam egestas at nisl non finibus. Mauris interdum elit lorem, sit amet eleifend ex rhoncus ultrices. Donec vulputate leo et velit viverra, sit amet volutpat nulla aliquam. Aenean quis velit sed arcu dapibus sagittis.

Etiam magna ipsum, eleifend ut egestas ac, varius ac mauris. Praesent non nisl vitae dui fermentum fermentum. Fusce a tortor vitae lorem semper scelerisque sagittis et magna. Aenean tincidunt, lacus nec lacinia tempus, felis urna placerat orci, at facilisis urna quam in nibh. Nam sagittis facilisis libero, non scelerisque leo finibus ac. Nulla neque purus, viverra sit amet elementum vel, iaculis in dui. Praesent sit amet condimentum est. Proin convallis sapien sed semper interdum. Mauris semper laoreet elementum. Nam eu fringilla odio, a rhoncus sapien. Etiam id sem quis sapien bibendum porttitor. Etiam eu luctus odio. Aenean vel imperdiet est, et dignissim mauris. Donec quam massa, accumsan eget tellus quis, accumsan dignissim enim. 

\section{Conclusão}
\label{sec:Conclusao}
Este trabalho apresentou...

Lorem ipsum dolor sit amet, consectetur adipiscing elit. Nunc eu mi cursus, pretium lectus vel, commodo neque. Nulla facilisi. Duis in quam non metus lobortis sagittis. Nunc ac auctor mi. Nullam imperdiet orci eget neque accumsan, eget ornare turpis cursus. Sed ut cursus sapien, vel accumsan neque. Donec dignissim maximus sapien non commodo. Praesent a gravida metus. Duis eget nulla luctus, finibus ex sed, ornare velit. Pellentesque habitant morbi tristique senectus et netus et malesuada fames ac turpis egestas. Curabitur placerat efficitur velit, eget auctor eros posuere nec. Morbi pretium purus in libero porttitor interdum.

%{\bf Acknowledgments}
%[Blind Review]

%\newcommand{\BIBdecl}{\setlength{\itemsep}{-0.5 em}}
%\bibliographystyle{IEEEtran}
\bibliographystyle{sbgames}
\bibliography{bibliography}

\end{document}
